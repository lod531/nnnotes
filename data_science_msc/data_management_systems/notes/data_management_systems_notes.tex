\documentclass{article}
\usepackage[utf8]{inputenc}
\usepackage[english]{babel}
\usepackage{setspace}
\usepackage{amssymb}
\usepackage{amsmath}
\usepackage{chngcntr}
\usepackage{float}
\usepackage{tabu}
\usepackage{bm}
\usepackage[lite]{amsrefs}
\usepackage{amsthm}
\usepackage{graphicx}
\usepackage{hyperref}\usepackage{xcolor}
%\graphicspath{ {./img/} }

\usepackage{graphicx}
\graphicspath{ {./images/} }

\usepackage{tikz}
\usetikzlibrary{matrix}


\newcommand{\shrug}[1][]{%
\begin{tikzpicture}[baseline,x=0.8\ht\strutbox,y=0.8\ht\strutbox,line width=0.125ex,#1]
\def\arm{(-2.5,0.95) to (-2,0.95) (-1.9,1) to (-1.5,0) (-1.35,0) to (-0.8,0)};
\draw \arm;
\draw[xscale=-1] \arm;
\def\headpart{(0.6,0) arc[start angle=-40, end angle=40,x radius=0.6,y radius=0.8]};
\draw \headpart;
\draw[xscale=-1] \headpart;
\def\eye{(-0.075,0.15) .. controls (0.02,0) .. (0.075,-0.15)};
\draw[shift={(-0.3,0.8)}] \eye;
\draw[shift={(0,0.85)}] \eye;
% draw mouth
\draw (-0.1,0.2) to [out=15,in=-100] (0.4,0.95); 
\end{tikzpicture}}




\counterwithin*{equation}{section}

\newcommand{\R}{\mathbb{R}}

\makeatletter
\newcommand*\bigcdot{\mathpalette\bigcdot@{1}}
\newcommand*\bigcdot@[2]{\mathbin{\vcenter{\hbox{\scalebox{#2}{$\m@th#1\bullet$}}}}}
\makeatother

\usepackage{afterpage}

\newcommand\blankpage{%
    \null
    \thispagestyle{empty}%
    \addtocounter{page}{-1}%
    \newpage}
    
\newtheorem{theorem}{Theorem}[section]
\newtheorem{definition}[theorem]{Definition}
\newtheorem{observation}[theorem]{Observation}
\newtheorem{corollary}{Corollary}[theorem]
\newtheorem{lemma}[theorem]{Lemma}

%\setlength{\parindent}{0pt}

\DeclareMathOperator*{\argmax}{\arg\!\max}
\DeclareMathOperator*{\argmin}{\arg\!\min}

\newcommand*{\defeq}{\mathrel{\vcenter{\baselineskip0.5ex \lineskiplimit0pt
                     \hbox{\scriptsize.}\hbox{\scriptsize.}}}%
                     =}

\usepackage[]{algorithm2e}


\newcommand*\OR{\ |\ }

\begin{document}


\title{Data Management Systems}
\author{}
\date{}

\maketitle

\section{Intro}

	\subsection{Physical and logical independence}
		
		User does not care about how the data is stored, and what format it's stored in memory - is it on RAM, is it NTFS, whatever.
		
		Logical independence means the data is the data, and You can have different views of it. 
		
		I guess physical independence is nuts and bolts of actual storage, logical is how the data is presented to the user.
	
	\subsection{Query Optimization}
	
		Equivalent results may be achieved by different operators, and some are more efficient than others e.g. join and select v.s. select and join.		
		
	\subsection{Data Integrity}
	
		Enforcement of legal values, so the database is intact/coherent.
	
	\subsection{Access Control}
	
	\subsection{Concurrency Control}
	
	\subsection{Recovery}
	
	\subsection{Data vs Query shipping}
	
	\subsection{Cons/Pros of Shared-Nothing}
	
	\subsection{Shared Memory}
	
	\subsection{Shared Disk}
	
	
\newpage
\section{Storage Systems}
	
	a

		
		
		
		
		
		
		
		
		
\end{document}