\documentclass{article}
\usepackage[utf8]{inputenc}
\usepackage[english]{babel}
\usepackage{setspace}
\usepackage{amssymb}
\usepackage{amsmath}
\usepackage{chngcntr}
\usepackage{float}
\usepackage{tabu}
\usepackage{bm}
\usepackage[lite]{amsrefs}
\usepackage{amsthm}
\usepackage{graphicx}
\usepackage{hyperref}\usepackage{xcolor}
%\graphicspath{ {./img/} }


\counterwithin*{equation}{section}

\newcommand{\R}{\mathbb{R}}

\makeatletter
\newcommand*\bigcdot{\mathpalette\bigcdot@{1}}
\newcommand*\bigcdot@[2]{\mathbin{\vcenter{\hbox{\scalebox{#2}{$\m@th#1\bullet$}}}}}
\makeatother

\usepackage{afterpage}

\newcommand\blankpage{%
    \null
    \thispagestyle{empty}%
    \addtocounter{page}{-1}%
    \newpage}
    
\newtheorem{theorem}{Theorem}[section]
\newtheorem{definition}[theorem]{Definition}
\newtheorem{observation}[theorem]{Observation}
\newtheorem{corollary}{Corollary}[theorem]
\newtheorem{lemma}[theorem]{Lemma}

%\setlength{\parindent}{0pt}

\DeclareMathOperator*{\argmax}{\arg\!\max}
\DeclareMathOperator*{\argmin}{\arg\!\min}

\newcommand*{\defeq}{\mathrel{\vcenter{\baselineskip0.5ex \lineskiplimit0pt
                     \hbox{\scriptsize.}\hbox{\scriptsize.}}}%
                     =}

\begin{document}

	\section{Old Statement}
	
		I do not think that the use of macaques in biomedical research can be ethically justified.
		
		To me, equal interests should receive equal consideration, regardless of colour of skin, gender (and lack thereof) or species. Macaques have an interest in not being experimented on (which is an assumption, but to me a reasonable one). They cannot waive this interest, as humans can by providing consent.
	
		The counter argument I can foresee is that the research could save many
	lives. If the research is indeed that important, I do not see why humans would not volunteer. If the research is not worth volunteering for, it is not worth disregarding the interests of macaques. Experimenting on macaques because humans failed to volunteer would signal to me that the research is not worth the sacrifice, doubly so when the sacrifice forced upon another creature without their consent.

	\section{Reflection}
	
		\subsection{Ethics}
		
			Statement is made that Macaques have interests, which could be construed as a scientific statement (behavioural science). First of all, "interests" are not defined so that's not ideal, but it was meant to describe interests in terms of welling - it is in the interest of a macaque to have food, and preferences - macaques are interested in bananas. This seems readily defendable. 
			
			No appeals to emotion have been made that I can see, nor appeals to religion, nor appeal to norms.
			
		\subsection{Normative Ethics}
		
			\textbf{a)} No virtues are referenced.
			
			\textbf{b) }There is a \textit{vague} deontological argument - that equal interests should be considered equally, or rather entities with interests have a right to have equal interests be considered equally. It's really more of a utilitarian/consequentialist argument however.
			
			\textbf{c)} The consequentialist argument in there is that as a consequence of animal experimentation, the interests of macaques are being ignored, and that's a bad consequence - but as a consequentialist argument it is kind of confused and weak, since one could make the argument that the consequences of overriding macaque interests are better since they may cure diseases affecting millions.
			
			However this is addressed by pointing out that humans are not volunteering to be experimented upon. There is a counter-counter argument that hinges on value-of-life etc.
			
			As for a normative theory, it wants to be in the rights camp but is using the language of a utilitarian theory Instead of doing the utilitarian thing and saying "well hey, let's override these interests of the few for the benefit of the many" the argument instead aims for "interests cannot be overridden without good reason", which is more rights-based, and so the argument should be talking about rights, not interests.
			
		\subsection{Moral Arguments}
			
			\textit{Can you detect any factual or normative premises in your standpoint?}
		
			There is only one factual premise - that macaques have certain comparable interests to our own interests. The normative premise is that equal interests should be given equal consideration.
			
			\textit{Are there any missing premises}		
			
			Yes! The premise that macaques and humans have a \textit{similar} interest in not being experimented upon is missing. 
			
			\textit{Are there any reasons provided why stated counterarguments are not valid?}
			
			The counter argument presented was utilitarian - sacrificing few for the good of many. It's valid so long as utilitarianism is accepted as the moral framework.
			
			\textit{Are there any analogies? }
			
			None that I can see.
			
			\textit{Can you detect any problems (e.g. inconsistences, contradictions) or flaws (e.g. a naturalistic
fallacy) in your argumentation? }

			No flaws, but a premise could be clearer and/or defended - do macaques and humans have a similar interest in not being experimented upon? If this were addressed, I think the argument would be stronger.
			
		\subsection{What is moral status?}

			\textit{According to your personal standpoint, are the macaques to be considered morally? Do
researchers have indirect or direct duties towards macaques? }		
			
			Yes, according to the old argument the researchers have direct duties to the macaques.

			\textit{Is there a relevant criterion/feature/characteristic (moral individualism) or a specific
relationship (moral relationalism) indicated for the moral consideration of macaques?}
			
			The relevant moral criterion in the old argument is the presence of interests, so squarely in the moral individualism category.
			
			\textit{What about the moral significance/weight/importance (hierarchical or egalitarian view) of the
macaques?}

			The old argument gives equal weight to equal interests, so not a hierarchical approach, unless one decides to interpret it as more interests yields greater consideration.
			
			\textit{Hospital fire}
			
			When it comes to the hospital fire experiment I think it's consistent - not many people would agree, but I think that the human animals don't have more claim to being rescued than the non-human animals.
			
		\subsection{Moral Individualism and Moral Relationalism}
		
			\textit{Can you identify any similarities or analogies in your written standpoint to one or several of the
ethical perspectives treated in the course (moral individualism, or moral relationalism)? }

			The old argument is aiming and failing to take an animal rights perspective. It's using Peter Singer's utilitarianism but attempting to make an animal-rights based conclusion.
			
			\textit{Which of the treated ethical perspectives would fit best with your personal standpoint? Why?}
			
			Currently, Tom Regan's rights-based approach is most closely aligned with my own beliefs, since I do not believe that the interests of animals can be overriden for the benefit of humans. I also quite like the trust and  betrayal perspective - I agree that We should not be betraying animals as We currently are.
			
	\section{Conclusions for revising my personal standpoint}
	
		Aside from a missing premise (which missing premise seems defendable with some elbow grease), the old argument is basically confused. It wants to be proclaiming animal rights, but lacks understanding of what rights are. It's using utilitarianist ideas and then attempts to arrive at animal rights.


	
\end{document} 